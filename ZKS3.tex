\documentclass{VUMIFPSkursinis}
\usepackage{algorithmicx}
\usepackage{algorithm}
\usepackage{algpseudocode}
\usepackage{amsfonts}
\usepackage{float}
\usepackage{amsmath}
\usepackage{bm}
\usepackage{caption}
\usepackage{color}
\usepackage{float}
\usepackage{graphicx}
\usepackage{listings}
\usepackage{subfig}
\usepackage{ltablex}
\usepackage{longtable}
\usepackage{wrapfig}
\usepackage{enumitem}
\usepackage{subfig}
\usepackage{pbox}
\renewcommand{\labelenumii}{\theenumii}
\renewcommand{\theenumii}{\theenumi.\arabic{enumii}.}
\renewcommand{\labelenumiii}{\theenumiii}
\renewcommand{\theenumiii}{\theenumii\arabic{enumiii}.}
% Titulinio aprašas
\university{Vilniaus universitetas}
\faculty{Matematikos ir informatikos fakultetas}
\department{Programų sistemų katedra}
\papertype{Projektinis darbas}
\title{Internetinio banko tinklalapis}
\titleineng{Bankininkystė}
\status{3 kurso 3 grupės studentai}
\author{Justas Tvarijonas}
\secondauthor{Džiugas Mažulis}   
\thirdauthor{Michal Stankevič}   
\supervisor{Kristina Lapin, Doc., Dr.}
\date{Vilnius – \the\year}

\begin{document}
\maketitle
\sectionnonum{Anotacija}
\subsectionnonum{Darbo tikslas}
Įvertinti 3-ių sukurtų maketų panaudojamumą ir pateikti išvadas, kaip toliau tęsti projektą. Šiam tikslui įgyvendinti išsikėlėme šiuos uždavinius:
\begin{enumerate}
	\item Kiekvienam maketui atlikti euristinį tikrinimą,
	\item Kiekvienam maketui atlikti pažintinę peržvalgą,
	\item Nustatyti tolimesnę projekto eigos kryptį.
\end{enumerate}
\subsectionnonum{Darbo pasiskirstymas}
\begin{itemize}
	\item Justas Tvarijonas - tvarijonasjustas@gmail.com
	\item Džiugas Mažulis - džiugas.mažulis@gmail.com
	\item Michal Stankevič - michal.stankevic@gmail.com 
\end{itemize}
\tableofcontents
\sectionnonum{Įvadas}
\subsectionnonum{Dalykinė sritis}
Internetinė bankininkystė.
\subsectionnonum{Probleminė sritis}
Vartotojo grafinės sąsajos išmokstamumo gerinimas, pagrindinėms funkcijoms pasiekti atliekamų žingsnių bei klaidų mažinimas.
\subsectionnonum{Naudotojai}
Banko klientas - turi galimybę atlikti mokėjimus, peržiūrėti sąskaitos išrašus, kurti mokėjimo ruošinius, ieškoti reikalingos informacijos.
\subsectionnonum{Darbo pagrindas}
Trečiojo laboratorinio darbo reikalavimai.
\section{Džiugo Mažulio maketo vertinimas}
\subsection{Euristinis tikrinimas}
\subsection{Pažintinė peržvalga}
\subsection{Apibendrinimas}
\section{Justo Tvarijono maketo vertinimas}
\subsection{Euristinis tikrinimas}
\subsection{Pažintinė peržvalga}
\subsection{Apibendrinimas}
\section{Michal Stankevič maketo vertinimas}
\subsection{Euristinis tikrinimas}
\subsection{Pažintinė peržvalga}
\subsection{Apibendrinimas}
\section{Išvados}
\end{document}
