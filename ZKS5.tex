\documentclass[oneside]{VUMIFPSkursinis}
\usepackage{algorithmicx}
\usepackage{algorithm}
\usepackage{algpseudocode}
\usepackage{amsfonts}
\usepackage{float}
\usepackage{amsmath}
\usepackage{bm}
\usepackage{caption}
\usepackage[hidelinks]{hyperref}
\usepackage{color}
\usepackage{float}
\usepackage{graphicx}
\usepackage{listings}
\usepackage{subfig}
\usepackage{ltablex}
\usepackage{longtable}
\usepackage{wrapfig}
\usepackage{enumitem}
\usepackage{subfig}
\usepackage{caption}
\usepackage{pbox}
\renewcommand{\labelenumii}{\theenumii}
\renewcommand{\theenumii}{\theenumi.\arabic{enumii}.}
\renewcommand{\labelenumiii}{\theenumiii}
\renewcommand{\theenumiii}{\theenumii\arabic{enumiii}.}
% Titulinio aprašas
\university{Vilniaus universitetas}
\faculty{Matematikos ir informatikos fakultetas}
\department{Programų sistemų katedra}
\papertype{Projektinis darbas}
\title{Internetinio banko tinklalapis}
\titleineng{Maketų analitiniai vertinimai}
\status{3 kurso 3 grupės studentai}
\author{Justas Tvarijonas}
\secondauthor{Džiugas Mažulis}   
\thirdauthor{Michal Stankevič}   
\supervisor{Kristina Lapin, Doc., Dr.}
\date{Vilnius – \the\year}
\bibliography{bibliografija}
\begin{document}
\maketitle
\sectionnonum{Anotacija}
\subsectionnonum{Darbo tikslas}
Įvertinti 3-ių sukurtų maketų panaudojamumą ir pateikti išvadas, kaip toliau tęsti projektą. Šiam tikslui įgyvendinti išsikėlėme šiuos uždavinius:
\begin{enumerate}
	\item Kiekvienam maketui atlikti euristinį tikrinimą,
	\item Kiekvienam maketui atlikti pažintinę peržvalgą,
	\item Nustatyti tolimesnę projekto eigos kryptį.
\end{enumerate}
\subsectionnonum{Darbo pasiskirstymas}
\begin{itemize}
	\item Justas Tvarijonas - tvarijonasjustas@gmail.com
	\item Džiugas Mažulis - džiugas.mažulis@gmail.com
	\item Michal Stankevič - michal.stankevic@gmail.com 
\end{itemize}
\tableofcontents
\section{Santrauka}
\section{Įvadas}
\subsectionnonum{Prototipas}
Prototipe įgyvendintas internetinės bankininkystės sistema, šis prototipas yra minimalistinis, kad būtų patogesnis mažai ir vidutiniškai patyrusiems IT vartotojams, įgyvendintos svarbiausios bei dažniausiai naudojamos banko funkcijos.
\subsectionnonum{Prototipo kūrimo priemonė}
Prototipas kurtas su "Axure" programine įranga, tai pakankamai sudėtinga, tačiau galinga prototipavimo priemonė, kurią suprasti padėjo dėstytojos pateikti mokomieji vaizdo įrašai.
\subsectionnonum{Dalykinė sritis}
Internetinė bankininkystė.
\subsectionnonum{Probleminė sritis}
Vartotojo grafinės sąsajos išmokstamumo gerinimas, pagrindinėms funkcijoms pasiekti atliekamų žingsnių bei klaidų mažinimas.
\subsectionnonum{Naudotojai}
Banko klientas - turi galimybę atlikti mokėjimus, peržiūrėti sąskaitos išrašus, kurti mokėjimo ruošinius, ieškoti reikalingos informacijos.
\subsectionnonum{Darbo pagrindas}
Penktojo laboratorinio darbo reikalavimai.
\section{Testavimo aprašas}
\subsection{Testuojamos užduotys}
\begin{enumerate}
	\item Pagrindiniame lange paslėpti sąskaitos informaciją.
	\item Atlikti paiešką apie mokėjimus.
	\item Paieškos lange išfiltruoti pagal naujienas.
	\item Nueiti iki vietinių mokėjimų.
	\item Pasirinkti mokėjimą Džiugui Mažuliui iš ankstesnių mokėjimų.
	\item Nustatyti sumą ir mokėjimo paskirtį.
	\item Pereiti į kitą mokėjimo žingsnį.
	\item Pakeisti mokėjimo sumą ir atlikti mokėjimą.
	\item Atsidaryti mokėjimo išrašų langą.
	\item Nustatyti išrašą nuo 2018-11-13.
\end{enumerate}
Matavimai:
\begin{itemize}
	\item Kiek laiko užtruko atliktį užduotį.
	\item Klaidų skaičius.
	\item Sustojimų skaičius.
	\item Pagalbos prašymų skaičius.
\end{itemize}
\subsection{Metodas}
Stebėjimai testuojant su naudotojais. Vartotojai užduotis atlieka vieni. Prieš testavimą vartotojas gaus klausimyną, kuris nustatis vartotojo charakteristika, tada bus pateikta internetinė sąsaja. Iš užduočių sąrašo prašysime vartotojo vykdyti užduotis po vieną ir stebėsime vartotojo sėkmę. Vartotojas bet kuriuo metu galės prašyti pagalbos. 
\subsection{Aplinka}
Dalis testavimo vykdoma gyvai, dalis nuotoliniu būdu,naudojant discord programinę įrangą. Abiejais atvejais vartotojo ekranas ir atliekami judesiai buvo stebima mūsų.
\subsection{Dalyviai}
\begin{center}
	\captionof{table}{Dalyvių lentelė}
	\begin{tabular}{ |p{1cm} | p{2cm} | p{4cm} | p{4cm} |}
	\hline
ID&Amžius&Profesija&IT raštingumas \\ \hline
1&-&-&- \\ \hline
2&-&-&- \\ \hline
\end{tabular}
\end{center}

\section{Testavimo rezultatai}
\subsection{Užduočių vykdymo rezultatai}
\begin{center}
	\begin{table}[!pht]
	\caption{Pirmo dalyvio lentelė}
	\begin{tabular}{ |p{1.8cm} | p{3.4cm} | p{3.4cm} | p{2.5cm} | p{3.5cm}|}
	\hline
	Užduotis&Padarytos klaidos&Sustojimų skaičius&Užtrukta laiko&Pagalbos prašymai\\ \hline
1.&-&-&-&- \\ \hline
2.&-&-&-&- \\ \hline
3.&-&-&-&- \\ \hline
4.&-&-&-&- \\ \hline
5.&-&-&-&- \\ \hline
6.&-&-&-&- \\ \hline
7.&-&-&-&- \\ \hline
8.&-&-&-&- \\ \hline
9.&-&-&-&- \\ \hline
10.&-&-&-&- \\ \hline
11.&-&-&-&- \\ \hline
12.&-&-&-&- \\ \hline
\end{tabular}
\end{table}
\vspace{0.7cm}
	\begin{table}[!pht]
	\caption{Pirmo dalyvio lentelė}
	\begin{tabular}{ |p{1.8cm} | p{3.4cm} | p{3.4cm} | p{2.5cm} | p{3.5cm}|}
	\hline
	Užduotis&Padarytos klaidos&Sustojimų skaičius&Užtrukta laiko&Pagalbos prašymai\\ \hline
1.&-&-&-&- \\ \hline
2.&-&-&-&- \\ \hline
3.&-&-&-&- \\ \hline
4.&-&-&-&- \\ \hline
5.&-&-&-&- \\ \hline
6.&-&-&-&- \\ \hline
7.&-&-&-&- \\ \hline
8.&-&-&-&- \\ \hline
9.&-&-&-&- \\ \hline
10.&-&-&-&- \\ \hline
11.&-&-&-&- \\ \hline
12.&-&-&-&- \\ \hline
\end{tabular}
\end{table}
\vspace{0.7cm}
	\begin{table}[!pht]
	\caption{Pirmo dalyvio lentelė}
	\begin{tabular}{ |p{1.8cm} | p{3.4cm} | p{3.4cm} | p{2.5cm} | p{3.5cm}|}
	\hline
	Užduotis&Padarytos klaidos&Sustojimų skaičius&Užtrukta laiko&Pagalbos prašymai\\ \hline
1.&-&-&-&- \\ \hline
2.&-&-&-&- \\ \hline
3.&-&-&-&- \\ \hline
4.&-&-&-&- \\ \hline
5.&-&-&-&- \\ \hline
6.&-&-&-&- \\ \hline
7.&-&-&-&- \\ \hline
8.&-&-&-&- \\ \hline
9.&-&-&-&- \\ \hline
10.&-&-&-&- \\ \hline
11.&-&-&-&- \\ \hline
12.&-&-&-&- \\ \hline
\end{tabular}
\end{table}
\vspace{0.7cm}
	\begin{table}[!pht]
	\caption{Pirmo dalyvio lentelė}
	\begin{tabular}{ |p{1.8cm} | p{3.4cm} | p{3.4cm} | p{2.5cm} | p{3.5cm}|}
	\hline
	Užduotis&Padarytos klaidos&Sustojimų skaičius&Užtrukta laiko&Pagalbos prašymai\\ \hline
1.&-&-&-&- \\ \hline
2.&-&-&-&- \\ \hline
3.&-&-&-&- \\ \hline
4.&-&-&-&- \\ \hline
5.&-&-&-&- \\ \hline
6.&-&-&-&- \\ \hline
7.&-&-&-&- \\ \hline
8.&-&-&-&- \\ \hline
9.&-&-&-&- \\ \hline
10.&-&-&-&- \\ \hline
11.&-&-&-&- \\ \hline
12.&-&-&-&- \\ \hline
\end{tabular}
\end{table}
\vspace{0.7cm}
	\begin{table}[!pht]
	\caption{Pirmo dalyvio lentelė}
	\begin{tabular}{ |p{1.8cm} | p{3.4cm} | p{3.4cm} | p{2.5cm} | p{3.5cm}|}
	\hline
	Užduotis&Padarytos klaidos&Sustojimų skaičius&Užtrukta laiko&Pagalbos prašymai\\ \hline
1.&-&-&-&- \\ \hline
2.&-&-&-&- \\ \hline
3.&-&-&-&- \\ \hline
4.&-&-&-&- \\ \hline
5.&-&-&-&- \\ \hline
6.&-&-&-&- \\ \hline
7.&-&-&-&- \\ \hline
8.&-&-&-&- \\ \hline
9.&-&-&-&- \\ \hline
10.&-&-&-&- \\ \hline
11.&-&-&-&- \\ \hline
12.&-&-&-&- \\ \hline
\end{tabular}
\end{table}
\vspace{0.7cm}
	\begin{table}[!pht]
	\caption{Pirmo dalyvio lentelė}
	\begin{tabular}{ |p{1.8cm} | p{3.4cm} | p{3.4cm} | p{2.5cm} | p{3.5cm}|}
	\hline
	Užduotis&Padarytos klaidos&Sustojimų skaičius&Užtrukta laiko&Pagalbos prašymai\\ \hline
1.&-&-&-&- \\ \hline
2.&-&-&-&- \\ \hline
3.&-&-&-&- \\ \hline
4.&-&-&-&- \\ \hline
5.&-&-&-&- \\ \hline
6.&-&-&-&- \\ \hline
7.&-&-&-&- \\ \hline
8.&-&-&-&- \\ \hline
9.&-&-&-&- \\ \hline
10.&-&-&-&- \\ \hline
11.&-&-&-&- \\ \hline
12.&-&-&-&- \\ \hline
\end{tabular}
\end{table}
\vspace{0.7cm}
	\begin{table}[!pht]
	\caption{Pirmo dalyvio lentelė}
	\begin{tabular}{ |p{1.8cm} | p{3.4cm} | p{3.4cm} | p{2.5cm} | p{3.5cm}|}
	\hline
	Užduotis&Padarytos klaidos&Sustojimų skaičius&Užtrukta laiko&Pagalbos prašymai\\ \hline
1.&-&-&-&- \\ \hline
2.&-&-&-&- \\ \hline
3.&-&-&-&- \\ \hline
4.&-&-&-&- \\ \hline
5.&-&-&-&- \\ \hline
6.&-&-&-&- \\ \hline
7.&-&-&-&- \\ \hline
8.&-&-&-&- \\ \hline
9.&-&-&-&- \\ \hline
10.&-&-&-&- \\ \hline
11.&-&-&-&- \\ \hline
12.&-&-&-&- \\ \hline
\end{tabular}
\end{table}
\vspace{0.7cm}
\end{center}
\subsection{Dalyvių komentarai}
\begin{center}
	\captionof{table}{Dalyvių komentarai}
	\begin{tabular}{ |p{3cm}| p{12cm} |}
	\hline
Užduoties Nr.&Komentarai\\ \hline
	\end{tabular}
\end{center}
\subsection{Apibendrinimas}
\subsection{Panaudojamumo matai ir siekiai}
\begin{center}
	captionof{table}{Panaudojamumo matų ir siekių patikrinimas.}
    \begin{tabular}{ |p{12cm}| p{3cm} |}
    \hline
	Siekis & Ar įvykdytas? \\ \hline
	Vartotojui įvedus ne mažiau, kaip 3 simbolius į paieškos langą bus siūlomi pilni paieškos tekstai. & Taip. \\ \hline
	Vartotojas galės įvesti sumą bei pradėti pavedimą pagrindiniame lange. & Taip. \\ \hline
	Mokėjimo išrašo datos apsibrėžime nebus būtina pasirinkti dieną (užteks mėnesio). & - \\ \hline
	Vartotojas galės pasirinkti gavėją iš gavėjų arba ankstesnių gavėjų sąrašų. & - \\ \hline
	Paieškos rezultatus bus galima filtruoti pagal bent 4 skirtingas kategorijas. & Taip. \\ \hline
	Vartotojas paieškos langą surasti gebės ne per ilgesnį laiką, nei 30 sekundžių. & Taip. \\ \hline
	Vartotojas gebės atlikti pavedimą per ne ilgiau kaip 2 minutes. & Taip. \\ \hline
	97\% vartotojų gebės pirmu bandymu rasti mokėjimo pavedimo funkciją. & - \\ \hline
	Vartotojas mokėjimų išrašo langą gebės pasiekti ne per daugiau, nei 2 mygtuko paspaudimus. & Taip. \\ \hline
	80\% vartotojų gebės parinkti tinkamą raktinį paieškos žodį norimiems rezultatams gauti. & - \\ \hline
	Vartotojas, rinkdamasis gavėją iš pateiktų sąrašų, gebės atlikti pavedimą 3 mygtukų paspaudimu. & - \\ \hline
	95\% vartotojų gebės grįžti iš mokėjimo patvirtinimo lango ir pakeisti mokėjiumo informaciją. & Taip. \\ \hline

    \end{tabular}
\end{center}
\section{Rekomendacijos}
\section{Priedai}
\subsection{Klausimynai}
\subsubsection{Klausimynas prieš testavimą}
\begin{itemize}
	\item Kaip vertinate savo IT žinias?
	\item Kiek laiko jau naudojatės internetine pankininkyste?
	\item Kiek valandų per dieną paprastai naudojatės kompiuteriu?
\end{itemize}
\subsubsection{Klausimynas po testavimo}
	\begin{itemize}
			\item Kas labiausiai patiko prototipe?
			\item Kas labiausiai nepatiko?
			\item Kaip vertinate sistemą bendrai?
			\item Ar susidūrėte su sunkumais, netikėtumais?
	\end{itemize}
\subsection{Dalyvių rezultatų lentelės}
\subsection{dalyvio sutikimo dalyvauti testavime raštas}
\end{document}
