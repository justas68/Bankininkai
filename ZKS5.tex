\documentclass[oneside]{VUMIFPSkursinis}
\usepackage{algorithmicx}
\usepackage{algorithm}
\usepackage{algpseudocode}
\usepackage{amsfonts}
\usepackage{float}
\usepackage{amsmath}
\usepackage{bm}
\usepackage{caption}
\usepackage[hidelinks]{hyperref}
\usepackage{color}
\usepackage{float}
\usepackage{graphicx}
\usepackage{listings}
\usepackage{subfig}
\usepackage{ltablex}
\usepackage{longtable}
\usepackage{wrapfig}
\usepackage{enumitem}
\usepackage{subfig}
\usepackage{caption}
\usepackage{pbox}
\renewcommand{\labelenumii}{\theenumii}
\renewcommand{\theenumii}{\theenumi.\arabic{enumii}.}
\renewcommand{\labelenumiii}{\theenumiii}
\renewcommand{\theenumiii}{\theenumii\arabic{enumiii}.}
% Titulinio aprašas
\university{Vilniaus universitetas}
\faculty{Matematikos ir informatikos fakultetas}
\department{Programų sistemų katedra}
\papertype{Projektinis darbas}
\title{Internetinio banko tinklalapis}
\titleineng{Maketų analitiniai vertinimai}
\status{3 kurso 3 grupės studentai}
\author{Justas Tvarijonas}
\secondauthor{Džiugas Mažulis}   
\thirdauthor{Michal Stankevič}   
\supervisor{Kristina Lapin, Doc., Dr.}
\date{Vilnius – \the\year}
\bibliography{bibliografija}
\begin{document}
\maketitle
\sectionnonum{Anotacija}
\subsectionnonum{Darbo tikslas}
Įvertinti 3-ių sukurtų maketų panaudojamumą ir pateikti išvadas, kaip toliau tęsti projektą. Šiam tikslui įgyvendinti išsikėlėme šiuos uždavinius:
\begin{enumerate}
	\item Kiekvienam maketui atlikti euristinį tikrinimą,
	\item Kiekvienam maketui atlikti pažintinę peržvalgą,
	\item Nustatyti tolimesnę projekto eigos kryptį.
\end{enumerate}
\subsectionnonum{Darbo pasiskirstymas}
\begin{itemize}
	\item Justas Tvarijonas - tvarijonasjustas@gmail.com
	\item Džiugas Mažulis - džiugas.mažulis@gmail.com
	\item Michal Stankevič - michal.stankevic@gmail.com 
\end{itemize}
\tableofcontents
\section{Santrauka}
Šio prototipo testavimo vykdytojų pagrindą sudarė studentai, tačiau buvo ir keletas prototipo kūrėjų šeimos narių. Dalyviai buvo patys paprašyti įsivertinti savo IT žinias bei gebėjimus, kadangi kitiems tai įvertinti yra pakankamai sudėtinga. Dalyvių pagrindą sudarė vidutinių ir gerų IT gebėjimų žmonės (šio produkto pagrindiniai naudotojai), bei keletas save įsiverinę kaip ekspertus. Visi dalyviai turi jau nemažą patirtį naudojant elektronine bankininkystę, kas jiem suteikia papildomų žinių tikrinant šį prototipą. \par
Prieš testavimą kiekvienas dalyvis gavo klausimyną jo charakteristikoms atskleisti, o po testavimo buvo pateiktas kitas klausimynas, kurio paskirtis buvo apibendrinti ir sustruktūrizuoti vertintojo nuomonę ir pastebėjimus apie tikrintą prototipą. Pats testavimas vyko keliose skirtingose aplinkose (dalis dalyvių testavo prie MIF patalpose esančių kompiuterių, dalis savo namuose), tačiau mūsų nuomone, testavimo rezultatų tai neįtakojo. \par
Testavimui buvo pasirinktas mąstymo garsiai metodas, tačiau net ir vykdytojui nutilus, jis nebuvo raginamas kalbėti pateikiant klausimus. \par
Testavimo metu buvo fiksuojami vertintojo žingsniai, skaičiuojamos metrikos, kurios, visiems dalyviams atlikus testavimą buvo apibendrintos ir sudėtos į atitinkamas lenteles grupuojant pagal užduočių numerius.
\section{Įvadas}
\subsectionnonum{Prototipas}
Prototipe įgyvendintas internetinės bankininkystės sistema, šis prototipas yra minimalistinis, kad būtų patogesnis mažai ir vidutiniškai patyrusiems IT vartotojams, įgyvendintos svarbiausios bei dažniausiai naudojamos banko funkcijos.
\subsectionnonum{Prototipo kūrimo priemonė}
Prototipas kurtas su "Axure" programine įranga, tai pakankamai sudėtinga, tačiau galinga prototipavimo priemonė, kurią suprasti padėjo dėstytojos pateikti mokomieji vaizdo įrašai.
\subsectionnonum{Dalykinė sritis}
Internetinė bankininkystė.
\subsectionnonum{Probleminė sritis}
Vartotojo grafinės sąsajos išmokstamumo gerinimas, pagrindinėms funkcijoms pasiekti atliekamų žingsnių bei klaidų mažinimas.
\subsectionnonum{Naudotojai}
Banko klientas - turi galimybę atlikti mokėjimus, peržiūrėti sąskaitos išrašus, kurti mokėjimo ruošinius, ieškoti reikalingos informacijos.
\subsectionnonum{Darbo pagrindas}
Penktojo laboratorinio darbo reikalavimai.
\section{Testavimo aprašas}
\subsection{Testuojamos užduotys}
\begin{enumerate}
	\item Pagrindiniame lange paslėpti sąskaitos informaciją. Sekmės kriterijus: atlikimo laikas mažiau, negu 10 sekundžių, klaidų, pagalbos prašymų 0.
	\item Atlikti paiešką apie mokėjimus. Sekmės kriterijus: atlikimo laikas mažiau, negu 30 sekundžių, klaidų skaičius ne daugiau 2, pagalbos pagalbos prašymų ne daugiau 1.
	\item Paieškos lange išfiltruoti pagal naujienas. Sekmės kriterijus: atlikimo laikas mažiau, negu 15 sekundžių, klaidų, pagalbos prašymų 0.
	\item Nueiti iki vietinių mokėjimų. Sekmės kriterijus: atlikimo laikas mažiau, negu 30 sekundžių, klaidų ne daugiau 1.  pagalbos prašymų 0.
	\item Pasirinkti mokėjimą Džiugui Mažuliui iš ankstesnių mokėjimų. Sekmės kriterijus: atlikimo laikas mažiau, negu 10 sekundžių, klaidų, pagalbos prašymų 0.
	\item Nustatyti sumą ir mokėjimo paskirtį. Sekmės kriterijus: atlikimo laikas mažiau, negu 25 sekundės, klaidų, pagalbos prašymų 0.
	\item Pereiti į kitą mokėjimo žingsnį. Sekmės kriterijus: atlikimo laikas mažiau, negu 10 sekundžių, klaidų, pagalbos prašymų 0.
	\item Pakeisti mokėjimo sumą ir valiutą, bei atlikti mokėjimą. Sekmės kriterijus: atlikimo laikas mažiau, negu 40 sekundžių, klaidų ne daugiau 1, pagalbos prašymų 0.
	\item Atsidaryti mokėjimo išrašų langą. Sekmės kriterijus: atlikimo laikas mažiau, negu 15 sekundžių, klaidų, pagalbos prašymų 0.
	\item Nustatyti išrašą nuo 2018-09-10. Sekmės kriterijus: atlikimo laikas mažiau, negu 30 sekundžių, klaidų, pagalbos prašymų 0.
	\item Pridėtį mokėjimo ruošinį (nuo pagrindinio lango). Sekmės kriterijus: atlikimo laikas mažiau, negu 80 sekundžių, klaidų ne daugiau 2 pagalbos prašymų ne daugiau 1.
	\item Atlikti mokėjimą pagal ruošinį (nuo pagrindinio lango). Sekmės kriterijus: atlikimo laikas mažiau, negu 60 sekundžių, klaidų ne daugiau 1, pagalbos prašymų 0.
	\item Atlikti pavedimą nesinaudojant meniu juosta(iš pagrindinio lango). Sekmės kriterijus: atlikimo laikas mažiau, negu 40 sekundžių, klaidų ne daugiau 1, pagalbos prašymų 0.
\end{enumerate}
Matavimai:
\begin{itemize}
	\item Kiek laiko užtruko atliktį užduotį.
	\item Klaidų skaičius.
	\item Sustojimų skaičius.
	\item Pagalbos prašymų skaičius.
\end{itemize}
\subsection{Metodas}
Stebėjimai testuojant su naudotojais. Vartotojai užduotis atlieka vieni, tačiau jų buvo paprašyta atliekant užduotis karsiai komentuoti jų atliekamus veiksmus. Prieš testavimą vartotojas gaus klausimyną, kuris nustatis vartotojo charakteristika, tada bus pateikta internetinė sąsaja. Iš užduočių sąrašo prašysime vartotojo vykdyti užduotis po vieną ir stebėsime vartotojo sėkmę. Vartotojas bet kuriuo metu galės prašyti pagalbos. 
\subsection{Aplinka}
Dalis testavimo vykdoma gyvai, dalis nuotoliniu būdu,naudojant discord programinę įrangą. Abiejais atvejais vartotojo ekranas ir atliekami judesiai buvo stebima mūsų.
\subsection{Dalyviai}
\begin{center}
	\captionof{table}{Dalyvių lentelė}
	\begin{tabular}{ |p{1cm} | p{3cm} | p{5cm} | p{5cm} |}
	\hline
ID&IT raštingumas & Naudojimosi el. bankininkyste patirtis & Valandos per dieną prie kompiuterio\\ \hline
1&Geras&4 metai&8 \\ \hline
2&Ekspertas&4 metai&10 \\ \hline
3&Ekspertas&5 metai&8 \\ \hline
4&Vidutinis&4 metai&2 \\ \hline
5&Vidutinis&6 metai&7 \\ \hline
\end{tabular}
\end{center}

\section{Testavimo rezultatai}
\subsection{Užduočių vykdymo rezultatai}
\begin{center}
	\begin{table}[!pht]
	\caption{Pirmo dalyvio lentelė}
	\begin{tabular}{ |p{1.8cm} | p{3.4cm} | p{3.4cm} | p{2.5cm} | p{3.5cm}|}
	\hline
	Užduotis&Padarytos klaidos&Sustojimų skaičius&Užtrukta laiko(sec)&Pagalbos prašymai\\ \hline
1.&0&0&4&0 \\ \hline
2.&0&1&10&0 \\ \hline
3.&0&0&3&0 \\ \hline
4.&0&0&5&0 \\ \hline
5.&0&0&4&0 \\ \hline
6.&0&0&15&0 \\ \hline
7.&0&0&2&0 \\ \hline
8.&0&0&13&0 \\ \hline
9.&2&1&21&0 \\ \hline
10.&0&0&18&0 \\ \hline
11.&2&2&70&1 \\ \hline
12.&0&0&20&0 \\ \hline
13.&1&0&15&0 \\ \hline
\end{tabular}
\end{table}
\vspace{0.7cm}
	\begin{table}[!pht]
	\caption{Antro dalyvio lentelė}
	\begin{tabular}{ |p{1.8cm} | p{3.4cm} | p{3.4cm} | p{2.5cm} | p{3.5cm}|}
	\hline
	Užduotis&Padarytos klaidos&Sustojimų skaičius&Užtrukta laiko&Pagalbos prašymai\\ \hline
1.& 0 & 0 &2&0 \\ \hline
2.&0&0&8&0 \\ \hline
3.&0&0&1&0 \\ \hline
4.&0&0&5&0 \\ \hline
5.&0&0&4&0 \\ \hline
6.&0&0&7&0 \\ \hline
7.&0&0&1&0 \\ \hline
8.&0&0&13&0 \\ \hline
9.&0&0&1&0 \\ \hline
10.&0&0&10&0 \\ \hline
11.&1&1&30&1 \\ \hline
12.&0&2&40&0 \\ \hline
13.&0&1&20&0 \\ \hline
\end{tabular}
\end{table}
\vspace{0.7cm}
	\begin{table}[!pht]
	\caption{Trečio dalyvio lentelė}
	\begin{tabular}{ |p{1.8cm} | p{3.4cm} | p{3.4cm} | p{2.5cm} | p{3.5cm}|}
	\hline
	Užduotis&Padarytos klaidos&Sustojimų skaičius&Užtrukta laiko&Pagalbos prašymai\\ \hline
1.&0&0&2&0 \\ \hline
2.&0&0&5&0 \\ \hline
3.&0&0&2&0 \\ \hline
4.&0&0&5&0 \\ \hline
5.&0&2&10&1 \\ \hline
6.&0&0&7&0 \\ \hline
7.&0&0&3&0 \\ \hline
8.&0&0&18&0 \\ \hline
9.&0&0&3&0 \\ \hline
10.&0&0&8&0 \\ \hline
11.&2&0&32&0 \\ \hline
12.&0&0&15&0 \\ \hline
13.&0&0&12&0 \\ \hline
\end{tabular}
\end{table}
\vspace{0.7cm}
	\begin{table}[!pht]
	\caption{Ketvirto dalyvio lentelė}
	\begin{tabular}{ |p{1.8cm} | p{3.4cm} | p{3.4cm} | p{2.5cm} | p{3.5cm}|}
	\hline
	Užduotis&Padarytos klaidos&Sustojimų skaičius&Užtrukta laiko&Pagalbos prašymai\\ \hline
1.&0&0&5&0 \\ \hline
2.&0&0&10&0 \\ \hline
3.&0&0&8&0 \\ \hline
4.&0&0&10&0 \\ \hline
5.&0&0&9&0 \\ \hline
6.&0&0&12&0 \\ \hline
7.&0&0&5&0 \\ \hline
8.&0&0&20&0 \\ \hline
9.&1&0&8&0 \\ \hline
10.&0&0&10&0 \\ \hline
11.&2&2&56&0 \\ \hline
12.&0&1&23&0 \\ \hline
13.&0&1&16&0 \\ \hline
\end{tabular}
\end{table}
\vspace{0.7cm}
	\begin{table}[!pht]
	\caption{Penkto dalyvio lentelė}
	\begin{tabular}{ |p{1.8cm} | p{3.4cm} | p{3.4cm} | p{2.5cm} | p{3.5cm}|}
	\hline
	Užduotis&Padarytos klaidos&Sustojimų skaičius&Užtrukta laiko&Pagalbos prašymai\\ \hline
1.&0&0&2&0 \\ \hline
2.&0&0&7&0 \\ \hline
3.&0&2&10&0 \\ \hline
4.&0&0&4&0 \\ \hline
5.&0&1&10&0 \\ \hline
6.&0&0&6&0 \\ \hline
7.&0&0&13&0 \\ \hline
8.&0&0&8&0 \\ \hline
9.&0&0&2&0 \\ \hline
10.&0&0&2&0 \\ \hline
11.&1&0&22&0 \\ \hline
12.&0&0&12&0 \\ \hline
\end{tabular}
\end{table}
\vspace{0.7cm}
\end{center}
\subsection{Dalyvių komentarai}
\begin{center}
	\captionof{table}{Dalyvių komentarai}
	\begin{tabular}{ |p{3cm}| p{12cm} |}
	\hline
Užduoties Nr.&Komentarai\\ \hline
2 & "Niekas šiaip paieška vistiek nesinaudoja". \\ \hline
5 & "Per daug čia tų pasirinkimų". \\ \hline
9 & Galima pagalvoti, kad išrašas yra mokėjimų punkte. \\ \hline
10 & Galutinė data turėtų būti automatiškai nustatyta šiandiena. \\ \hline
11 & Klaidinantys meniu pasirinkimai. Neaišku, kad ruošinius kuriame prie \\ \hline
13 & "Neiškarto supratau ką čia padaryt reikia". \\ \hline
	\end{tabular}
\end{center}
\subsection{Apibendrinimas}
\begin{center}
	\begin{tabular}{ |p{1.6cm}| p{2cm} |p{3cm}|p{3cm}|p{3cm}| p{2.4cm} |}
\hline
Užduoties Nr.&Sėkmingai įvykdė&Vykdymo laiko vidurkis (s)&Kiek kartų kreiptąsi pagalbos&Padarytos klaidos&Sustojimų skaičius\\ \hline
1  &  100\% & 3 & 0 & 0 & 0\\ \hline
2  &  100\% & 8 & 0 & 0 & 1 \\ \hline
3  &  100\% & 4.8 & 0 & 0 & 2 \\ \hline
4  &  100\% &  5.8 & 0 & 0 & 0\\ \hline
5  &  80\% &  6.2 & 1 & 0 & 2 \\ \hline
6  &  100\% &  9.4 & 0 & 0 & 0 \\ \hline
7  &  80\% & 4.8 & 0 & 0 & 0 \\ \hline
8  &  100\% & 14.4 & 0 & 0 & 0 \\ \hline
9  &  60\% & 7 & 0 & 3 & 1\\ \hline
10 &  100\% & 9.6 & 0 & 0 & 0 \\ \hline
11 &  80\% &  42 & 3 & 7 & 5\\ \hline
12 &  80\% & 24 & 0 & 1 & 3\\ \hline
13 &  100\% & 15 & 0 & 1 & 2\\ \hline
\end{tabular}
\end{center}
\newpage
\subsection{Panaudojamumo matai ir siekiai}
\begin{center}
	\begin{table}[!ht]
	\caption{Panaudojamumo matų ir siekių patikrinimas.}
    \begin{tabular}{ |p{12cm}| p{3cm} |}
    \hline
	Siekis & Ar įvykdytas? \\ \hline
	Vartotojui įvedus ne mažiau, kaip 3 simbolius į paieškos langą bus siūlomi pilni paieškos tekstai. & Taip. \\ \hline
	Vartotojas galės įvesti sumą bei pradėti pavedimą pagrindiniame lange. & Taip. \\ \hline
	Mokėjimo išrašo datos apsibrėžime nebus būtina pasirinkti dieną (užteks mėnesio). & Taip. \\ \hline
	Vartotojas galės pasirinkti gavėją iš gavėjų arba ankstesnių gavėjų sąrašų. & Taip. \\ \hline
	Paieškos rezultatus bus galima filtruoti pagal bent 4 skirtingas kategorijas. & Taip. \\ \hline
	Vartotojas paieškos langą surasti gebės ne per ilgesnį laiką, nei 30 sekundžių. & Taip. \\ \hline
	Vartotojas gebės atlikti pavedimą per ne ilgiau kaip 2 minutes. & Taip. \\ \hline
	97\% vartotojų gebės pirmu bandymu rasti mokėjimo pavedimo funkciją. & Taip \\ \hline
	Vartotojas mokėjimų išrašo langą gebės pasiekti ne per daugiau, nei 2 mygtuko paspaudimus. & Taip. \\ \hline
	80\% vartotojų gebės parinkti tinkamą raktinį paieškos žodį norimiems rezultatams gauti. & Taip. \\ \hline
	Vartotojas, rinkdamasis gavėją iš pateiktų sąrašų, gebės atlikti pavedimą 3 mygtukų paspaudimu. & Ne. \\ \hline
	95\% vartotojų gebės grįžti iš mokėjimo patvirtinimo lango ir pakeisti mokėjimo informaciją. & Taip. \\ \hline
	90\% žmonių gebės pasinaudoti mokėjimo ruošiniu atlikdami mokėjimą. & Taip. \\ \hline
	80\% žmonių gebės susikurti mokėjo ruošinį. & Taip. \\ \hline

		\end{tabular}
	\end{table}
\end{center}
\section{Rekomendacijos}

\begin{itemize}
	\item Dažnis - Skaičius vartotojų, kurie susidūrė su šiuo defektu.
	\item Prioritetas:
		\begin{itemize}
			\item Mažas - nedidelis dažniausiai visualinis defektas, netrukdantis darbui.
			\item Vidutinis - defektas, trukdantis darbui, tačiau per daug neįtakojantis klaidų skaičiaus.
			\item Didelis - defektas, kuris dažnai sukelia klaidas.
		\end{itemize}
\end{itemize}
\begin{center}
    \begin{tabular}{ |p{2cm}| p{4cm} | p{2cm} | p{2cm} | p{5cm} |}
    \hline
	Užduoties Nr.&Defektas&Dažnis&Prioritetas&Siūlomas sprendimas\\ \hline
	5&Per daug skirtukų prie ruošinių &2&Vidutinis&Suparasti ruošinių lentelę, paliekant 2 esminius stulpelius(vienas ankstesnių mokėjimų, kitas vartotojo sukurtų ruošinių). \\ \hline
	9&Galima pagalvoti, kad mokėjimo išrašų langas turi būti mokęjimų lange.&2&Vidutinis&Vienas iš sprendimų pridėti papildomą nuorodą į išrašo langą iš mokėjimų lango, kadangi vietos trūkumo tam padaryti nėra.\\ \hline
	10&Galutinė data nenustatyta į šiandieną.&4&Mažas&Pakeisti galutinę datą taip, kad jis visada būtų einamoji diena. \\ \hline
	11&Ruošinių sukūrimas priskirtas prie vietinių mokėjimų.&5&Didelis& Perkelti mokėjimo ruošinių sukūrimą į jam priskirtą langą, iš kurio nueiti galima iš mokėjimo lango, tuo pačiu supaprastinant vietinių mokėjimo langą. \\ \hline
    \end{tabular}
\end{center}

\section{Priedai}
\subsection{Klausimynai}
\subsubsection{Klausimynas prieš testavimą}
\begin{itemize}
	\item Kaip vertinate savo IT žinias?
	\item Kiek laiko jau naudojatės internetine pankininkyste?
	\item Kiek valandų per dieną paprastai naudojatės kompiuteriu?
\end{itemize}
\subsubsection{Klausimynas po testavimo}
	\begin{itemize}
			\item Kas labiausiai patiko prototipe?
			\item Kas labiausiai nepatiko?
			\item Kaip vertinate sistemą bendrai?
			\item Ar susidūrėte su sunkumais, netikėtumais?
	\end{itemize}
	\newpage
\subsection{Dalyvių rezultatų lentelės}
\begin{center}
	\begin{tabular}{ |p{1cm} | p{3cm} | p{4cm} | p{4cm} | p{3cm} |}
	\hline
ID&Kas labiausiai patiko?&Kas labiausiai nepatiko?&Kaip vertinate sistemą bendrai?&Ar susidūrėte su sunkumais, netikėtumais?\\ \hline
1&Paveiksliukai, aišku kur rasti kategorijas.&Iš naujo atidarant mokėjimų langą, duomenys iš praeito mokėjimo neturėtų būti įvesti.&Patogu naudotis, bet neišbaigta.&Taip, ruošinio sukūrimas bei išrašo pasirinkimas neaiškūs. \\ \hline
2&Simplstiškas dizainas, ikonos, kurios reprezentuoja savo funkcijas.&Nepatinka naujienų skiltis, žinučių skiltis.(palikt daugiau erdvės) Vietinių mokėjimų lange visko per daug.&7/10, Sistema atlieka savo funkcionalumą, estetiškai graži, tačiau nemažai vietų perkrautos nereikalingais UI elementais. & Nesusidūriau, kaip ekspertui, viskas buvo sąlyginai aišku. \\ \hline
3&Vienareikšmiški pasirinkimai.&Išrašų lange neišbaigtas dizainas.& Trūksta išbaigtumo, bet naudotis patogu.&Gavėjas Džiugas Mažulis nurodytas ne prie ankstesnių gavėjų, o prie visų, kas neatitinka vienos iš užduočių. \\ \hline
4&Graži išvaizda&Nepatiko mokėjimų ruošiniai.&Gerai, paliko teigiamą įspūdį.& Neimplementuotas ruošinių pasiekimas iš mokėjimų lango glumina vartotoją.\\ \hline
5&Intuityvus meniu.&Ruošiniai.&Neišdirbta, bet perspektyvi.&Ruošiniai nepasiekiami standartinėje vietoje.\\ \hline
\end{tabular}
\end{center}
\newpage
\subsection{dalyvio sutikimo dalyvauti testavime raštas}
\begin{figure}[ht]
	\centering
	\includegraphics[width=16cm,height=70cm,keepaspectratio]{Sutikimas.png}
	\caption{Sutikimo raštas}
\end{figure}
\end{document}
