\documentclass{VUMIFPSkursinis}
\usepackage{algorithmicx}
\usepackage{algorithm}
\usepackage{algpseudocode}
\usepackage{amsfonts}
\usepackage{amsmath}
\usepackage{bm}
\usepackage{caption}
\usepackage{color}
\usepackage{float}
\usepackage{graphicx}
\usepackage{listings}
\usepackage{subfig}
\usepackage{wrapfig}
\usepackage[utf8]{inputenc}
\usepackage[T1]{fontenc}  

% Titulinio aprašas
\university{Vilniaus universitetas}
\faculty{Matematikos ir informatikos fakultetas}
\department{Programų sistemų katedra}
\papertype{Projektinis darbas}
\title{Internetinio banko tinklalapis}
\titleineng{Website of internet bank}
\status{3 kurso 3 grupės studentai}
\author{Justas Tvarijonas}
\secondauthor{Džiugas Mažulis}   
\thirdauthor{Michal Stankiewicz}   
\supervisor{Kristina Lapin}
\date{Vilnius – \the\year}

\begin{document}
\maketitle

\tableofcontents
\sectionnonum{Anotacija}
Šiuo darbu siekiama išanalizuoti ir aprašyti dabartinės Swedbank sąsajos napatogumus, paaiškinti koks panaudojimo principas buvo pažeistas ir šio pažeidimo priežastis.
Šiame darbe taip pat atkreipsime dėmesį į sistemos vartotojų grupes ir jų sąveika su sitema.
Darbo eigoje apžvelgsime pasisekusias vartotojo sąsajos realizacijas ir aptarsime, kodėl būtent tokie sprendimai yra geresni už esamos sistemos sąsajų realizacijas.
\begin{itemize}
	\item Justas Tvarijonas - Tvarijonasjustas@gmail.com
	\item Džiugas Mažulis -
	\item Michal Stankiewicz -
\end{itemize}
\sectionnonum{Įvadas}
\subsectionnonum{Dalykinė sritis}
Internetinė bankininkystė, finansai.
\subsectionnonum{Probleminė sritis}
Vartotojų patogumo gerinimas, patogus informacijos pateikimas.
\subsectionnonum{Naudotojai}
\begin{itemize}
	\item Banko klientas - klientas turi galėti atlikti bankinius pervedimus, užsakymus, bei rasti norimą informaciją.
	\item ?
\end{itemize}
\subsectionnonum{Darbo pagrindas}
Pirmojo laboratorinio darbo reikalavimai.
\section{Būsimos sistemos įtakojamų asmenų kategorijos}
\subsection{Suinteresuotų asmenų grupės}
\begin{itemize}
	\item Pirminiai - Banko klientai.
	\item Antriniai - Banko darbuotojai, kurie žmonės aiškina, kaip naudotis sistema.
	\item Tretiniai - Kiti bankai, kurių klientų skaičių įtakoja šio banko sekmė, akcininkai, kurių pajamos priklauso nuo banko sekmės.
\end{itemize}
\section{Banko klientų poreikiai}
\subsection{Naudotojų charakteristikos}
\subsubsection{Informacinių technologijų priemonės}
\begin{itemize}
	\item Išmanusis telefonas (naršyklė bei smard-id).
	\item Mobilusis parašas.
	\item Asmeninių kompiuterių naršyklės.
	\item Planšetinių kompiuterių naršyklės.
\end{itemize}
\subsubsection{Motyvacija ir galimybės tobulinti įgūdžius}
\begin{itemize}
	\item Klientai turi skirtingus IT įgūdžius.
	\item Klientai skirtingais dažnumais naudojasi elektronine bankininkyste.
\end{itemize}
\subsubsection{Veiklų kontekstai}
\begin{itemize}
	\item Veikla yra pertraukiama. Klientas gali suformuluoti mokėjimą, tada užsiimti kita veikla ir vėliau grįžti užbaigti mokėjimą.
	\item Mokėjimai atliekami su dideliu susikaupimu.
	\item Naudojimasis sistema atliekamas saugioje aplinkoje.
	\item Paprastai sistema naudojami turint aiškų tikslą.
\end{itemize}
\subsubsection{Naudotojų tipas}
\subsection{Kompiuterizuojamų veiklų analizė}
\subsubsection{<pirmosios kompiuterizuojamos veiklos>? koncepcinis scenarijus}
\subsubsection{<pirmosios kompiuterizuojamos veiklos>? veiklų charakteristikos}
\subsubsection{<pirmojo scenarijaus> problemos ir tobulinimo galimybės}
\subsubsection{būsimasis patobulintas scenarijus}
\subsection{Panaudojamumo siekiai ir matai}
\section{Įkvepiančios esamų interfeisų idėjos}
\section{Terminų žodynėlis(maybe)}

\end{document}
