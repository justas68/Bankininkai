\documentclass{VUMIFPSkursinis}
\usepackage{algorithmicx}
\usepackage{algorithm}
\usepackage{algpseudocode}
\usepackage{amsfonts}
\usepackage{amsmath}
\usepackage{bm}
\usepackage{caption}
\usepackage{color}
\usepackage{float}
\usepackage{graphicx}
\usepackage{listings}
\usepackage{subfig}
\usepackage{wrapfig}
\usepackage[utf8]{inputenc}
\usepackage[T1]{fontenc}  

% Titulinio aprašas
\university{Vilniaus universitetas}
\faculty{Matematikos ir informatikos fakultetas}
\department{Programų sistemų katedra}
\papertype{Projektinis darbas}
\title{Internetinio banko tinklalapis}
\titleineng{Website of internet bank}
\status{3 kurso 3 grupės studentai}
\author{Justas Tvarijonas}
\secondauthor{Džiugas Mažulis}   
\thirdauthor{Michal Stankiewicz}   
% \fourthauthor{Vardonis Pavardonis}   % Pridėti ketvirtą autorių
\supervisor{Kristina Lapin}
\date{Vilnius – \the\year}

\begin{document}
\maketitle

\tableofcontents
\sectionnonum{Anotacija}
Šiuo darbu siekiama išanalizuoti ir aprašyti dabartinės Swedbank sąsajos napatogumus, paaiškinti koks panaudojimo principas buvo pažeistas ir šio pažeidimo priežastis.
Šiame darbe taip pat atkreipsime dėmesį į sistemos vartotojų grupes ir jų sąveika su sitema.
Darbo eigoje apžvelgsime pasisekusias vartotojo sąsajos realizacijas ir aptarsime, kodėl būtent tokie sprendimai yra geresni už esamos sistemos sąsajų realizacijas.
\begin{itemize}
	\item Justas Tvarijonas - Tvarijonasjustas@gmail.com
	\item Džiugas Mažulis - 
	\item Michal Stankiewicz - 
\end{itemize}

\sectionnonum{Įvadas}
	\subsectionnonum{Dalykinė sritis}
	  Internetinė bankininkystė, finansai.
  \subsectionnonum{Probleminė sritis}
    Vartotojų patogumo gerinimas, patogus informacijos pateikimas.
	\subsectionnonum{Naudotojai}
		\begin{itemize}
      \item Banko klientas - klientas turi galėti atlikti bankinius pervedimus, užsakymus, bei rasti norimą informaciją.
      \item ?
		\end{itemize}
\section{Būsimos sistemos įtakojamų asmenų kategorijos}
  \subsection{Suinteresuotų asmenų grupės}
    \begin{itemize}
      \item Pirminiai - Banko klientai bei darbuotojai aptarnaujantys žmones
      \item Antriniai - Banko savininkas, akcininkai(Užsakovai).
      \item Tretiniai - Kiti bankai, kurių klientų skaičių įtakoja šio banko sekmė. 
    \end{itemize}
\section{<Pirmos naudotojų grupės poreikiai}
  \subsection{Naudotojų charakteristikos}
  \subsection{Kompiuterizuojamų veiklų analizė}
  \subsection{Panaudojamumo siekiai ir matai}
\section{Įkvepiančios esamų interfeisų idėjos}
\section{Terminų žodynėlis(maybe)}

\end{document}
